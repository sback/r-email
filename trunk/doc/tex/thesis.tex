\documentclass[]{usiinfthesis}
\title{REmail} %compulsory 
\subtitle{Integrating e-mail communication in the Eclipse IDE} %optional 
\author{Vitezslav Humpa} %compulsory 
\advisor{MIchele Lanza} %compulsory 
\coadvisor{Alberto Bacchelli} %optional
\day{17} %defaults to \today 
\month{May} %compulsory 
\year{2010} %compulsory, put only the year 
\place{Lugano} %compulsory 
\programDirector{The PhD program Director \emph{pro tempore}} %compulsory
\committee{% 
%\committeeMember{Alonzo Church}{University of California, Los Angeles, USA} 
%\committeeMember{Alan M. Turing}{Princeton University, USA} %there can as many members as you like
} %the committee is compulsory
%\dedication{To my beloved} %optional 
%\openepigraph{Someone said \dots}{Someone} %optional
\makeindex %optional, also use \theindex at the end

\begin{document} 

\maketitle %generates the titlepage, this is FIXED 

\frontmatter %generates the frontmatter, this is FIXED

\begin{abstract}
Software system developers have to communicate about the project they are building. Especially when they work in open source projects and they are spread all over the world, developers use e-mails to communicate.

Studies tell us that e-mails are, by far, the most used means of communication used during the development, opposed to instant messaging, commits, or code comments. Therefore, we can imagine that development mailing lists contain essential information concerning various entities of the source code, that unfortunately get lost with time, since they are not easy to retrieve.

We have developed REmail, an Eclipse plugin that integrates e-mail communication into the the IDE. It allows developers to seamlessly handle source code entities and e-mails concerning the source code, without ever exiting from the IDE, using several techniques to provide the most appropriate linking between mailing list and software project in mind.

\end{abstract}

%\begin{abstract}[Zusammenfassung] %creates a new abstract section with "Zusammenfassung" as heading 

\begin{acknowledgements}
\end{acknowledgements}

\tableofcontents
%\listoffigures %optional 
%\listoftables %optional %add any other lists here
\setlength{\parskip}{\baselineskip}
\mainmatter
%------------------------------
\chapter{Introduction}

REmail is an Eclipse plugin we are developing that integrates e-mail communication in the IDE to get the most out of it. Using the current release, REmail allows a developer, without ever exiting from the Eclipse IDE, to select a Java class, automatically retrieve all the related messages from a specified e-mail archive, and read them in a convenient way. REmail is currently under heavy development, but the basic functionality of the plugin has been established.

As usually, introduction (since also it should point out the basic ideas of the following chapters and so... ) and conclusion to be written as almost the last...

\section{Motivation}

\chapter{Related work}
\section{Lightweight methods}

\chapter{REmail as an Eclipse plug-in}
Introduction to the chapter, short description of what this chapter shall offer.
\section{Eclipse}
%Reason why are choosing Eclipse for the plugin - choice over etc netbeans.
With the goal of putting the idea of e-mail integration into practice, we had to decide what IDE we would actually use. There are many choices, especially if we consider various development environments for different programming languages.

The idea was certainly to make REmail work in an IDE that can be used for developing systems in multiple languages. Therefore many environments, especially those centered on C/C++ put out of question. 

Eclipse structure as a collection of a bunch of plugins.

Important parts of Eclipse plugin development.

Plugin.xml

Plugin class - lazy loading in the startup and so on.

Eclipse plugins that are used by REmail.

\section{"User Manual"}

\subsection{Source formats}

Explanation of the basic differences, intro into this subsection.

\subsubsection{MBox}
Possibilities of getting an MBox file of the entire Mailing Lists.

Miler to obtain mailing lists

How to use Miler.

Future Improvement/integration of Miler into the REmail preferences or so.

\subsubsection{PostgreSQL}

Description of how the PostgreSQL can be used.

Some sort of explanation, that it's not really working that well now - viz. next section.

\subsection{Using REmail}
Introduction unto the real "User Manual"

\subsubsection{Instalation}

\subsubsection{Setting up}

\subsubsection{Searching}

\subsubsection{Browsing e-mails}

\subsubsection{Editor Integration}

\section{Implementing REmail}
Introduction into the story of implementation

Early stages as SDE project.

Ideas of using thunderbird

Idea to make a separate thunderbird plugin

Why we decided otherwise.

...

\chapter{Future}
\section{Points of improvement}
\section{Future goals}

\appendix %optional, use only if you have an appendix

\chapter{Some retarded material}
\section{It's over\dots}
%-------------------------------
\backmatter 
\chapter{Glossary} %optional
%\bibliographystyle{alpha} %any style compatible with the natbib package 
%\bibliographystyle{dcu} %\bibliographystyle{plainnat}
%\bibliography{biblio}
\cleardoublepage %the index starts on a recto! \theindex %optional, use only if you have an index, must use
%\makeindex in the preamble 
\end{document}